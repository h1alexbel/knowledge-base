% (The MIT License)
%
% Copyright (c) 2022 Aliaksei Bialiauski
%
% Permission is hereby granted, free of charge, to any person obtaining a copy
% of this software and associated documentation files (the 'Software'), to deal
% in the Software without restriction, including without limitation the rights
% to use, copy, modify, merge, publish, distribute, sublicense, and/or sell
% copies of the Software, and to permit persons to whom the Software is
% furnished to do so, subject to the following conditions:
%
% The above copyright notice and this permission notice shall be included in all
% copies or substantial portions of the Software.
%
% THE SOFTWARE IS PROVIDED 'AS IS', WITHOUT WARRANTY OF ANY KIND, EXPRESS OR
% IMPLIED, INCLUDING BUT NOT LIMITED TO THE WARRANTIES OF MERCHANTABILITY,
% FITNESS FOR A PARTICULAR PURPOSE AND NONINFRINGEMENT. IN NO EVENT SHALL THE
% AUTHORS OR COPYRIGHT HOLDERS BE LIABLE FOR ANY CLAIM, DAMAGES OR OTHER
% LIABILITY, WHETHER IN AN ACTION OF CONTRACT, TORT OR OTHERWISE, ARISING FROM,
% OUT OF OR IN CONNECTION WITH THE SOFTWARE OR THE USE OR OTHER DEALINGS IN THE
% SOFTWARE.

\documentclass[nobrand,anonymous,nodate,nosecurity]{huawei}
\usepackage{multicol}
\usepackage{href-ul}
\usepackage{ffcode}
\begin{document}

{\sffamily{\bfseries\Large Software Developer's Knowledge Base}\\
Papers by \href{h1alexbel.github.io/about-me.html}{Aliaksei Bialiauski}

\begin{abstract}
This is a series of lectures related to mainly JVM Backend Software Development.
It starts from basics like Git and UML and dives into Advanced Backend System Design and DevOps. The lectures provide basics and includes best practices for each topic.
\end{abstract}

\textbf{What is the goal?}\\
To unify knowledge and best practices for Java/JVM Software Developers in one place.
\section*{Knowledge Base Structure}

\begin{itemize}
\item Software Design: Engineering Requirements
\item Source control:
	\href{https://git-scm.com}{Git}
\item Software Design: Textual documentation:
	\href{https://en.wikipedia.org/wiki/Markdown}{Markdown},
	\href{https://en.wikipedia.org/wiki/Wiki}{Wiki},
	\href{https://en.wikipedia.org/wiki/LaTeX}{LaTeX}
\item OOP
\item FP
\item Design Patterns
\item UML
\item Java
\item Groovy
\item Kotlin
\item JVM Frameworks:
	\href{https://spring.io}{Spring},
	\href{https://projectreactor.io}{Project Reactor},
	\href{https://ktor.io}{Ktor}
\item RDBMS:
	\href{https://www.postgresql.org.pl}{PostgreSQL},
	\href{https://www.wikiwand.com/en/Object%E2%80%93relational_mapping}{ORM}
\item NoSQL:
	\href{https://www.mongodb.com}{MongoDB},
	\href{https://aws.amazon.com/dynamodb/}{DynamoDB},
	\href{https://cassandra.apache.org/_/index.html}{Cassandra}
\item Messaging:
	\href{https://kafka.apache.org}{Apache Kafka}
	\href{https://www.rabbitmq.com}{RabbitMQ}
\item Software Testing:
	\href{}{TDD},
	\href{}{BDD},
	\href{}{ATDD},
	\href{https://junit.org/}{JUnit},
	\href{https://www.testcontainers.org}{Testcontainers},
	\href{https://jmeter.apache.org}{JMeter},
	\href{https://site.mockito.org}{Mockito},
	\href{https://github.com/powermock/powermock}{PowerMock},
	\href{https://www.eclemma.org/jacoco/}{JaCoCo},
	\href{codecov.io/}{Codecov},
	\href{}{Mutation coverage}
\item Dependencies, Build automation, CI/CD:
	\href{https://en.wikipedia.org/wiki/Make_(software)}{Make},
	\href{https://maven.apache.org}{Maven},
	\href{https://gradle.org}{Gradle},
	\href{https://gradle.org}{GitHub Actions}
\item DevOps:
	\href{https://www.docker.com}{Docker},
	\href{https://kubernetes.io}{K8s},
	\href{https://www.heroku.com}{Heroku},
	\href{https://aws.amazon.com/}{AWS},
	\href{https://www.terraform.io}{Terraform}
\item Advanced System Design
\item Integration Development
\item Big Data
\item IoT
\end{itemize}

\newpage
\section*{Learning Material}

The following books are highly recommended to read (in no particular order):

\begin{multicols}{2}\small\raggedright
{Robert Martin}, \emph{Clean Architecture: A Craftsman's Guide to Software Structure and Design}\\[3pt]
{Robert C. Martin}, \emph{Clean Code: A Handbook of Agile Software Craftsmanship}\\[3pt]
{David Thomas et al.}, \emph{The Pragmatic Programmer: Your Journey To Mastery}\\[3pt]
{Michael Feathers}, \emph{Working Effectively with Legacy Code}\\[3pt]
{\nospell{Jez Humble} et al.}, \emph{Continuous Delivery: Reliable Software Releases through Build, Test, and Deployment Automation}\\[3pt]
{\nospell{Michael T. Nygard}}, \emph{Release It!: Design and Deploy Production-Ready Software}\\[3pt]
{\nospell{Saurabh Shrivastava}}, \emph{Solutions Architect's Handbook}\\[3pt]
{Martin Fowler}, \emph{UML Distilled}\\[3pt]
{\nospell{Martin Kleppmann}}, \emph{Designing Data-Intensive Applications}\\[3pt]
{\nospell{Neha Narkhede}}, \emph{Kafka: The Definitive Guide: Real-Time Data and Stream Processing at Scale}\\[3pt]
{\nospell{Yegor Bugayenko}}, \emph{Code Ahead}\\[3pt]
{Project Management Institute}, \emph{A Guide to the Project Management Body of Knowledge}\\[3pt]
{Martin Fowler}, \emph{Refactoring: Improving the Design of Existing Code}\\[3pt]
{\nospell{Yegor Bugayenko}}, \emph{Elegant Objects, Volume 1}\\[3pt]
{\nospell{Yegor Bugayenko}}, \emph{Elegant Objects, Volume 2}\\[3pt]
{Mark Richards, Neal Ford}, \emph{Fundamentals of Software Architecture: An Engineering Approach}\\[3pt]
{Kent Beck}, \emph{Test-Driven Development: By Example}\\[3pt]
{Steve Freeman, Nat Pryce}, \emph{Growing Object-Oriented Software, Guided by Tests}\\[3pt]
{Eric Evans}, \emph{Domain-Driven Design: Tackling Complexity in the Heart of Software}\\[3pt]
{Martin Fowler}, \emph{Patterns of Enterprise Application Architecture}\\[3pt]

\end{multicols}


\end{document}